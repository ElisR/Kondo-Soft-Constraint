% !TEX root = Final_Report.tex

\section{Theoretical Background} % Remember to fix the sigma subscripts
\label{sec:theory}

\subsection{Kondo Model}

The Kondo model made its first appearance in 1964 when theorists were attempting to explain puzzling experimental observations made 30 years earlier that certain metals containing magnetic impurities showed minima in their resistivity as a function of temperature. Jun Kondo proposed the Kondo model to describe a new scattering mechanism introduced by magnetic impurities which accounted for the functional form of the resistivity.

In its simplest single impurity flavour, the Kondo model has the following Hamiltonian: \begin{equation} H_{\text{Kondo}}=\sum_{\boldsymbol{k},\sigma}\epsilon_{\boldsymbol{k}} c_{\boldsymbol{k},\sigma}^{\dagger}c^{}_{\boldsymbol{k},\sigma}+J\vec{S}\cdot\vec{s}(0) , \label{eq:KondoHamiltonian}\end{equation} in which there is a kinetic energy contribution from the conduction electrons $ c^{}_{\boldsymbol{k}, \sigma}$ as well as a term describing an antiferromagnetic coupling between a spin localised at the origin and the spin density of conduction electrons at that point.\footnote{The spin density of conduction electrons is given by $$ \vec{s}(0) = \frac{1}{2} \sum_{\boldsymbol{k}, \boldsymbol{k'}} \sum_{\sigma, \sigma'} c^{\dagger}_{\boldsymbol{k}, \sigma} \vec{\tau}_{\sigma, \sigma'} c^{}_{\boldsymbol{k'}, \sigma'} ~, $$ where $ \vec{\tau}_{\sigma, \sigma'} $ is a vector of Pauli matrices.}
% Check the factor of 1/2 here


% Should probably include a discussion of the Kondo model and things we know about its behaviour
% Talk about Kondo resonance and the Kondo temperature
% Talk about how the Kondo model is a low temperature limit to the Anderson model
% Indeed one of the reasons for the Kondo model's historical importance is how it lead to the development of the renormalisation group

% Falls within class of strongly correlated systems so usual band theory may not be enough like what might be possible with semiconductors etc
% Strongly correlated: can't think of things in terms of one-particle excitations
% Great intellectual challenge and so new methods to be able to go some way in predicting new materials

\subsection{Mean-Field Theory}

The theoretical calculations of this project are framed in terms of the path integral, which is an approach to statistical mechanics reminiscent of Feynman's path integral formulation of quantum mechanics. The partition function can be written as the functional integral over fermionic paths: $$ Z = \Tr{e^{- \beta H}} = \int \mathcal{D} [c^\dagger, c]~e^{-\int_{0}^{\beta} \,d\tau~L}\,, $$ from which many properties of the system may then be derived.\footnote{One also replaces fermion operators with \emph{Grassman numbers} within the integral, which have the property of anti-commutation (among others).} Here, the equivalent \emph{action} involves integration of the Lagrangian $ L $ over an imaginary time $ \tau = i t / \hbar $ with an upper limit of $ \beta = \frac{1}{k_B T} $.

% Put the k_B footnote here

% Don't be as informal with the word 'tricks'
One then begins to employ many `tricks' to make the problem more tractable. One such trick that proves to be useful is the introduction of new boson operators, which sometimes (as we shall see) necessitates that hard constraints be applied in the form of Lagrange multipliers. The way that constraints are implemented into the Lagrangian is shown in Appendix~\ref{sec:Constraints}.

The essence of mean-field theory is that we avoid performing the actual functional integration by approximating the integral by its saddle point, a step also known as the stationary phase approximation. In making this approximation, we are essentially imposing self-consistency conditions on whatever fields now appear in $ L $, making them take on their mean values. Thankfully, these mean-field self-consistency equations are exactly what result from directly minimising the effective action with respect to the auxiliary fields, which is what this project will involve.

% Maybe talk about how mean-field theory can be used to study BCS theory etc
% Maybe mention extensions like Dynamical Mean-Field theory

\subsubsection{Kotliar-Ruckenstein Bosons}

\subsection{The Soft-Constraint Approach}

% Make the k into a vector?

One existing mean-field approach is that of Read and Newns \cite{ReadNewns}, which represents the $ \vec{S}_0 $ term of the Hamiltonian in Eq~\eqref{eq:KondoHamiltonian} in terms of slave fermions $ f^{}_{\sigma} $, with occupation numbers obeying the hard constraint $ \sum_{\sigma} f^{\dagger}_{\sigma} f^{}_{\sigma} = 1 \,. $ The new mean-field approach that is being proposed reformulates this constraint as: \begin{equation} (1 - n_{\uparrow} - n_{\downarrow})^2 = n_{\uparrow} n_{\downarrow} + (1 - n_{\uparrow})(1 - n_{\downarrow}) = 0 . \label{eq:HardConstraint}\end{equation} Implementing this into the Lagrangian is formally equivalent, but problematic within mean-field theory because one later imposes: \[ \langle (1 - n_{\uparrow} - n_{\downarrow})^2 \rangle = 0\,, \] which, for such a positive semi-definite operator, enforces the exact constraint and leads to a diverging mean-field parameter. (Appendix~\ref{sec:Divergence} gives some feeling for why this is the case.)

A resolution to this issue is to begin by introducing an auxiliary non-interacting fermion $ h $ that is constrained to be trivially empty through imposing $ h^{\dagger} h = 0 $. One then combines this constraint with that of Eq~\eqref{eq:HardConstraint} by imposing instead: \begin{equation} n_{\uparrow} n_{\downarrow} + (1 - n_{\uparrow})(1 - n_{\downarrow}) - K h^{\dagger} h = 0\,, \end{equation} such that $ K > 0, K \neq 1 $ now encapsulates both constraints. Notice that this has introduced an arbitrary parameter $ K $ into the problem and thus a new degree of freedom, but has allowed us to circumvent the issues related to the previous hard constraint within mean-field theory. For this reason, this approach to mean-field theory has been internally referred to as the \emph{soft constraint approach}.

Proceeding in a similar fashion to the Read-Newns approach and incorporating four Kotliar-Ruckenstein (KR) \cite{KotliarRuckenstein} slave bosons, one obtains the following Lagrangian:
\begin{align}
\begin{split}
L \quad = \quad &\sum_{\boldsymbol{k},\sigma} c^{\dagger}_{\boldsymbol{k},\sigma} \left( \frac{d}{\,d\tau} + \epsilon^{}_{\boldsymbol{k}} - \mu \right) c^{}_{\boldsymbol{k},\sigma} + \sum_{\sigma} f^{\dagger}_{\sigma} \frac{d}{\,d\tau} f^{}_{\sigma} + h^{\dagger} \frac{d}{\,d\tau}h \\
&+ e^{\dagger} \frac{d}{\,d\tau} e  + \sum_{\sigma} p^{\dagger}_{\sigma} \frac{d}{\,d\tau} p^{}_{\sigma} + d^{\dagger} \frac{d}{\,d\tau} d\\
&+ \sum_{\sigma} \lambda^{}_{\sigma} (f^{\dagger}_{\sigma} f^{}_{\sigma} - p^{\dagger}_{\sigma} p^{}_{\sigma} - d^{\dagger} d^{} ) + \lambda_{\text{KR}} ( e^{\dagger} e + \sum_{\sigma} p^{\dagger}_{\sigma} p^{}_{\sigma} + d^{\dagger} d - 1)\\
&+\lambda_{\text{SC}} ( e^{\dagger} e + d^{\dagger} d - K h^{\dagger} h)\\
&+ 2 \frac{V V^{\ast}}{J} + \sum_{\boldsymbol{k},\sigma} \left( V^{\ast} c^{\dagger}_{\boldsymbol{k},\sigma} z^{}_{\sigma} f^{}_{\sigma} + V f^{\dagger}_{\sigma} z^{\dagger}_{\sigma} c^{}_{\boldsymbol{k},\sigma} \right) \,.\label{eq:Lagrangian}
\end{split}
\end{align}
The slave bosons $ e $, $ p_{\uparrow} $, $ p_{\downarrow} $ and $ d $ represent empty, singly occupied and doubly occupied states, respectively, as long as the fermion operator is suitably transformed to `update the books', so to speak:\footnote{The conventional transformation $ z^{}_{\sigma} \rightarrow (1 - d^{\dagger} d - p^{\dagger}_{\sigma} p^{}_{\sigma})^{- 1 / 2} z^{}_{\sigma} (1 - e^{\dagger} e - p^{\dagger}_{- \sigma} p^{}_{- \sigma})^{- 1 / 2} $ is also applied, but is not relevant to this discussion.} \begin{equation} f^{}_{\sigma} \rightarrow z^{}_{\sigma} f^{}_{\sigma}\,, \qquad z_{\sigma} = e^{\dagger} p^{}_{\sigma} + p^{\dagger}_{-\sigma} d \,. \end{equation} A Hubbard-Stratonovich transformation has also been applied to remove certain unpleasant terms in the Lagrangian, framing the interaction in terms of a new bosonic field $ V $.

% Don't say that the conventional transformation is not relevant, mention the slave boson review i.e. that including this means that the U=0 Anderson model can be recovered
% In this representation we have chosen to quantize along the z-axis, though it is possible to use Kotliar-Ruckenstein slave bosons in a rotationally invariant way

% From here, one may then derive the Helmholtz free energy and other quantities of interest within mean-field theory. The freedom in the parameter $ K $ allows for a whole family of mean-field solutions, from which one may obtain a single solution by tuning $ K $ to fit a known property of the Kondo model, for instance the Kondo temperature $ T_K $. Having tuned this parameter, other predictions about the system are made possible.
