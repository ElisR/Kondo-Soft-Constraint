% Beamer Presentation

\documentclass[13pt]{beamer}
\usetheme{metropolis}
\usepackage{eulervm}
\usepackage{appendixnumberbeamer}
\usepackage{amsmath}

\DeclareMathOperator{\Tr}{Tr}

\setsansfont[BoldFont={Fira Sans}]{Fira Sans Light}
\setmonofont{Fira Mono}
\metroset{block=fill}

\title{Mean-Field Study of Kondo Phase Diagram}
\date{\today}
\author{Elis Roberts} % Should probably include supervisors
\institute{University of Cambridge}

\begin{document}
  \maketitle

  \begin{frame}{Plan}
  \setbeamertemplate{section in toc}[sections numbered]
  \tableofcontents[hideallsubsections]
  \end{frame}

  \section{Introduction}

  \begin{frame}{Kondo Model}
    Single localised spin interacting with conduction electrons $c^{}_{k,\sigma} $

    \vfill

    \begin{block}{Model Hamiltonian}
    $$ H_{\text{Kondo}}=\sum_{k,\sigma}\epsilon_{k} c_{k,\sigma}^{\dagger}c^{}_{k,\sigma}+J\vec{S}\cdot\vec{s}(0) $$
    \end{block}

  Can map spin \textsc{DOF}s of impurity onto fermions $ f^{}_{\sigma} $

  \end{frame}

  \begin{frame}{Path Integral $ \rightarrow $ Mean-Field Theory}

  \begin{itemize}
    \item Project framed in terms of path integral
    \item Treats Boltzmann sum as a functional integral over field configurations
  \end{itemize}

    \begin{block}{Many-Body Path Integral}
      $$ Z = \Tr{e^{- \beta H}} = \int \mathcal{D} [c^\dagger, c]~e^{-\int_{0}^{\beta} \,d\tau~L} $$
    \end{block}

  \begin{itemize}
    \item Similar to Feynman path integral approach, but with \textit{imaginary time} $ \tau = i t / \hbar $ and associated Lagrangian $ L $
  \end{itemize}

  \begin{block}{Mean-Field Theory}
    Avoid this difficult integral by making stationary phase approximation \hfill i.e. \textit{minimise}
  \end{block}

  \end{frame}

  \begin{frame}{New Fields}

  \begin{itemize}
    \item Lagrangian similar to original Hamiltonian, with extra dynamical terms: $ L = H_{\text{Kondo}} + c^\dagger \partial_\tau c$
    \item \textbf{But}, interaction term introduces difficult non-quadratic terms

    \begin{alertblock}{Resolution} % Make W boson analogy
      Introduce a new field $ V $ that mediates this interaction:
      \begin{figure}
        \centering
        \includegraphics[width=0.95\textwidth]{Figures/V_field.png}
      \end{figure}
    \end{alertblock}
  \end{itemize}

  \end{frame}

  \begin{frame}{Constraints}

  \begin{itemize}
    \item Making such variable changes sometimes requires \textbf{constraints}
  

  \begin{exampleblock}{e.g. Spin-1/2 $ \rightarrow $ Fermion Mapping}
  $$ s_z = \tfrac{1}{2} (f^{\dagger}_{\uparrow} f^{}_{\uparrow} - f^{\dagger}_{\downarrow} f^{}_{\downarrow}), \quad s_{+} = f^{\dagger}_{\uparrow} f^{}_{\downarrow}, \quad s_{-} = f^{\dagger}_{\downarrow} f^{}_{\uparrow} $$

  Representation isn't faithful unless we impose the constraint: $$ f^{\dagger}_{\uparrow} f^{}_{\uparrow} + f^{\dagger}_{\downarrow} f^{}_{\downarrow} = 1 $$
  \end{exampleblock}

    \item These constraints are implemented by adding Lagrange multipliers $ \{ \lambda_i \}$ and extremesing wrt these too - \textit{e.g.} $$ L \quad \supset \quad \lambda_{\text{RN}} (f^{\dagger}_{\uparrow} f^{}_{\uparrow} + f^{\dagger}_{\downarrow} f^{}_{\downarrow} - 1) $$

  \end{itemize}
  \end{frame}

  % Want to mention that there may be multiple ways to implement a constraint, but these differ in how they act at mean field level

  \begin{frame}{Soft-Constraint Approach}

  \begin{itemize}
    \item Now consider rewriting the same constraint: $$ (1 - n_{\uparrow} - n_{\downarrow})^2 = n_{\uparrow} n_{\downarrow} + (1 - n_{\uparrow})(1 - n_{\downarrow}) = 0 $$
    \item \textbf{But} this can't be done at mean-field level since $$ \langle (1 - n_{\uparrow} - n_{\downarrow})^2 \rangle \neq 0 $$

    \end{itemize}

    \begin{block}{Soft-Constraint}
      How about introducing a new fermion $ h $ obeying $ h^{\dagger} h^{} = 0 $, but combining these constraints by instead imposing $$ (1 - n_{\uparrow} - n_{\downarrow})^2 - K h^{\dagger} h^{} = 0 \quad ? $$
    \end{block}

  \end{frame}

  \begin{frame}{Soft-Constraint Approach}

  \begin{itemize}
    \item How could $ (1 - n_{\uparrow} - n_{\downarrow})^2 - K h^{\dagger} h^{} = 0 $ be any better?
    \item Have introduced a new free parameter $ K $ into description
    \item Provided $ K > 0 $ and $ K \neq 1 $, constraint operator now has positive \textit{and} negative eigenvalues: $$ \{ \quad 1, \quad (1 - K), \quad 0, \quad -K \quad \} $$
  \end{itemize}

  \begin{block}{Consequence}
  New composite constraint picks out both constraints, while stilll being possible at mean-field level
  \end{block}
  \end{frame}

  \section{Finite Temperature Study}

  \begin{frame}{Solving Mean-Field Equations}

  Generate MF equations via minimisation:
  $$ \frac{\partial F_{\text{MF}}}{\partial \Delta} = 0, \qquad \frac{\partial F_{\text{MF}}}{\partial \lambda_{\text{SC}}} = 0, \qquad \ldots $$

  \end{frame}

  \begin{frame}{Deriving Heat Capacity}
  \end{frame}

  \begin{frame}{Unavoidable Phase Transition?}
  \end{frame}

  \begin{frame}{Behaviour of Order Parameter $ \Delta $}
  \end{frame}

  \section{Finite Field Study}

  \begin{frame}{Incorporating $ B \neq 0 $}
  \end{frame}

  \begin{frame}{Increased Difficulty of Mean-Field Equations}
  \end{frame}

  \section{Conclusion}

  \begin{frame}[standout]
  Questions?
  \end{frame}

  \appendix

  \section{Backup}

  \begin{frame}{Final Lagrangian}
    \begin{equation}
    \begin{align*}
    L &= \sum_{k,\sigma} c^{\dagger}_{k,\sigma} \left( \frac{d}{\,d\tau} + \epsilon_k - \mu \right) c^{}_{k,\sigma} + \sum_{\sigma} f^{\dagger}_{\sigma} \frac{d}{\,d\tau} f^{}_{\sigma} + h^{\dagger} \frac{d}{\,d\tau}h \\
    &+ e^{\dagger} \frac{d}{\,d\tau} e  + \sum_{\sigma} p^{\dagger}_{\sigma} \frac{d}{\,d\tau} p^{}_{\sigma} + d^{\dagger} \frac{d}{\,d\tau} d \\
    &+ \sum_{\sigma} \lambda^{}_{\sigma} (f^{\dagger}_{\sigma} f^{}_{\sigma} - p^{\dagger}_{\sigma} p^{}_{\sigma} - d^{\dagger} d^{} ) \\
    &+ \lambda_{\text{KR}} ( e^{\dagger} e + \sum_{\sigma} p^{\dagger}_{\sigma} p^{}_{\sigma} + d^{\dagger} d - 1 ) + \lambda_{\text{SC}} ( e^{\dagger} e + d^{\dagger} d - K h^{\dagger} h) \\
    &+ 2 \frac{V V^{\ast}}{J} + \sum_{k,\sigma} \left( V^{\ast} c^{\dagger}_{k,\sigma} z^{}_{\sigma} f^{}_{\sigma} + V f^{\dagger}_{\sigma} z^{\dagger}_{\sigma} c^{}_{k,\sigma} \right)
    \end{align*}
    \end{equation}
  \end{frame}

\end{document}