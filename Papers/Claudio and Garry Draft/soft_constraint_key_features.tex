%% LyX 2.2.3 created this file.  For more info, see http://www.lyx.org/.
%% Do not edit unless you really know what you are doing.
\documentclass[twocolumn,english]{revtex4-1}
\usepackage[T1]{fontenc}
\usepackage[utf8]{luainputenc}
\setcounter{secnumdepth}{3}
\usepackage{amsmath}
\usepackage{esint}

\makeatletter
%%%%%%%%%%%%%%%%%%%%%%%%%%%%%% User specified LaTeX commands.
\usepackage{babel}

\makeatother

\usepackage{babel}
\begin{document}

\title{Mean-field method for handling null constraints on positive-definite
operators}

\author{Garry Goldstein$^{1}$, Claudio Chamon$^{2}$, Claudio Castelnovo$^{1}$}

\affiliation{$^{1}$TCM Group, Cavendish Laboratory, University of Cambridge,
J. J. Thomson Avenue, Cambridge CB3 0HE, United Kingdom}

\affiliation{$^{2}$Department of Physics, Boston University, Boston, Massachusetts
02215, USA}
\begin{abstract}
We present an approach for studying systems with hard constraints
that certain positive-definite operators must vanish. The difficulty
with mean-field treatments of such cases is that imposing that the
constraint is zero only in average is problematic for a quantity that
is always positive. We reformulate the hard constraints by adding
an auxiliary system such that some of the states to be projected out
from the total system are at finite negative energy and the rest at
finite positive energy. This auxiliary system comes with an extra
coupling that is unfixed, and that parametrizes a whole family of
mean-field theories. We argue that this variational-type parameter
for the family of mean-field theories should be fixed by matching
a given experimental observation, with the quality of the resulting
mean-field approximation measured by how it fits other data. We test
these ideas in the well-understood single-impurity Kondo problem,
where we fix the parameter via the $T_{K}$ obtained from the magnetic
susceptibilty value, and score the quality of the approximation by
its predicted Wilson ratio. 
\end{abstract}
\maketitle

\section{\label{sec: intro} Introduction }

One of the key theoretical challenges in condensed matter physics
is to understand systems with strong correlations, when interactions
become larger than the kinetic energy dispersion bandwidth. Physical
systems which fall in this important class include: the cuprate superconductors,
where strong interactions between electrons in copper's 3d shells
lead to an antiferromagnetic Mott insulator at half-filling and superconductivity
upon hole doping; heavy-electron compounds where localized d and f
orbitals of rare earth and transition metal atoms interact with itinerant
electrons, leading to a renormalization of the electron mass by as
much as a thousand times; and systems of cold atomic gases, where
interactions may be tuned by a Feshbach resonance to realize strongly
interacting multispecies systems \cite{key-2-3,key-3-2,key-1,key-2,key-3,key-4,key-5,key-9,key-10}.
In these and other examples of strongly correlated systems of interest
in modern condensed matter physics, it is customary to encounter a
broad range of energy scales that make any theoretical study a tall
order. One can often make substantial progress by producing effective
models where the largest energy scales are ``projected out''; namely,
the system is assumed to be in the lowest energy states of these large
energy terms, thus introducing strict conditions (or constraints)
in the effective models. These constraints lead to effective models
such as the t-J model, the Heisenberg model, and the Kondo model,
to name a few.

In this work we propose a method, which we call the soft constraint,
to study systems where the constraint is that a positive-definite
operator anihilates the physical states. A mean-field treatment of
the constraint, where it is satisfied in average, not identically,
is that the average is already lower-bounded by zero, because of the
positive-definiteness of the operator. Therefore, the mean-field theory
would be biased towards failure. Here we propose a method to circunvent
this problem: we introduce an auxiliary degree of freedom with its
own constraint, and construct a combined constraint that is satisfied
only when both the original and new constraints are satisfied. The
new quantity is no longer positive definite, and hence the combined
constraint can now be satisfied only in average. This method comes
with an additional parameter than can be varied freely. We thus have
not one, but instead a family of theories that, when treated exactly,
reproduce the correct constraints. When treated within a mean-field
approximation, the family of therories become different approximations
to the same physical problem. We propose to fix the value of the parameter
as follows: we choose one physical observable, and find the value
of the ``variational mean-field'' parameter to best match the observable.
The usefulness of the approximation with this fixed parameter can
be measured by how it fits other observables. We choose as a benchmark
the single-impurity Kondo problem, where we fix the parameter via
the $T_{K}$ obtained from the magnetic susceptibilty value, and grade
the quality of the approximation by the deviation of the predicted
Wilson ratio from the exact Bethe Ansatz result.

The introduction of the auxiliary system allows us to impose the zero-average
condition in the combined operator, but as we shall see, the combined
operator is often not quadratic. To deal with quartic terms, we use
the Kotliar-Ruckenstein approach. The combination of these two methods
produces better results then the conventional Read-Newns meanfield
for the Kondo model \cite{key-1} for the Wilson ratio, and it reproduces
the width of the Abrikosov-Suhl resonance to two loop level for the
Kondo RG \cite{key-1}.

In addition to the Kondo problem, we show that the method gives the
exact density of states for the infinite $U$ Anderson model. In fact,
we use this system as a simple way to first present, in section~\ref{app:constraints},
the basics of the soft-constraint method. Other systems that could
be studied using these techniques include hard-core bosons, spinor
condensates, t-J, Heisenberg, two-channel Kondo and two-channel Anderson
models.

\section{A first example: the infinite $U$ Anderson model}

\label{app:constraints}

As a first simple example of the soft constraint method we would like
to present a soft constraint formulation of the infinite $U$ Anderson
model focusing on the case where there is one spin $1/2$ impurity.
The Hamiltonian for the Anderson model is given by: 
\begin{equation}
H_{A}=Un_{\uparrow}n_{\downarrow}+\sum_{k,\sigma}\left(\epsilon_{k}-\mu\right)c_{k,\sigma}^{\dagger}c_{k,\sigma}+V\sum_{k,\sigma}c_{k,\sigma}^{\dagger}f_{\sigma}+h.c.\label{eq:Anderson Hamiltonian}
\end{equation}
Where the limit $U\rightarrow\infty$ is understood. We will consider
an auxiliary system unrelated to the impurity which is used to satisfy
the soft constraint. A variety of auxiliary systems are possible,
slave bosons, slave fermions, slave spins and even slave compound
systems; we shall focus on the case of the slave fermion $h$. Let
\begin{equation}
H=H_{A}+H'\label{eq:Auxilary_hamiltonian}
\end{equation}

With 
\begin{equation}
H'=\tilde{U}h^{\dagger}h\label{eq:Auxilary_space_hamiltonian}
\end{equation}

We will consider the case when $\tilde{U}\rightarrow\infty$. in this
case we are constraining all the fermions to lie in their ground state
so we are effectively multiplying \textbf{(CCh - ???)} the initial
Anderson model $H_{A}$ by one. We have two hard constrains in this
system: 
\begin{eqnarray}
h^{\dagger}h & = & 0\nonumber \\
n_{\uparrow}n_{\downarrow} & = & 0\label{eq:Constraints}
\end{eqnarray}

We would like to replace these with an equivalent constraint where
\begin{equation}
\tilde{\Pi}=n_{\uparrow}n_{\downarrow}-Kh^{\dagger}h=0\label{eq:effective_constraint}
\end{equation}

Here $K>0$ and $K\neq1$. As such the partition function of the Anderson
model is given by: 
\begin{eqnarray}
Z & = & e^{-\beta H_{A}}=\prod_{n}\left\{ e^{-\varepsilon_{n}\tilde{H}_{A}}\prod_{s}\tilde{\Pi}_{s}\right\} \nonumber \\
 & = & \frac{1}{N}\prod_{n}e^{-\varepsilon_{n}\tilde{H}_{A}}\prod_{s}\int d\lambda_{s,n}e^{-\varepsilon_{n}\lambda_{s,n}\left(n_{s,\sigma}n_{s,\sigma'}-Kh^{\dagger}h\right)}\label{eq:Anderson_model_partition_slave_fermion}
\end{eqnarray}

In this identity $K$ is an arbitrary parameter. We now take the stationary
phase with respect to the parameter $\lambda_{n}\rightarrow\lambda^{0}$.
In which case we get the equation: 
\begin{equation}
\frac{dZ}{d\lambda_{n}}=0\Rightarrow\left\langle n_{\uparrow}n_{\downarrow}-Kh^{\dagger}h\right\rangle =0\label{eq:Mean_field_slave_fermion}
\end{equation}

We would like to note that if we had not introduced the slave fermion
$h$ or equivalently $K=0$ we would not have obtained a solvable
meanfeld as $n_{\uparrow}n_{\downarrow}$ is a positive semidefinite
operator. For our next step we can maximize the partition function
with respect to $K$, from which we get that: 
\begin{equation}
\frac{dZ}{dK}=0\Rightarrow\lambda^{0}\left\langle h^{\dagger}h\right\rangle =0\label{eq:Mean_field_max_k}
\end{equation}

From which we get that the optimum meanfield (with the largest partition
function) is given by $\lambda^{0}=0$. This means that at meanfield
the partition function for the Anderson model is given by: 
\begin{equation}
Z=e^{-\beta H_{A}}=\prod_{n}\left\{ e^{-\varepsilon_{n}\tilde{H}_{A}}\prod_{s}\tilde{\Pi}_{s}\left(\lambda_{s}=0\right)\right\} =\frac{1}{N}\prod_{n}e^{-\varepsilon_{n}\tilde{H}_{A}}\label{eq:Anderson_model_meanfield_slave_fermion}
\end{equation}

That is its the partition function of the model with $U=0$. For the
case of the single impurity with energy $E_{f}$ it is known that
the physics of the non-interacting model is controlled by the physics
of the single impurity (with no free fermions) where the action is
given by: 
\begin{equation}
S_{A}=\sum_{n}\sum_{\sigma}\left(-i\omega_{n}+E_{f}-i\Delta sgn\left(\omega_{n}\right)\right)f_{\sigma}^{\dagger}f_{\sigma}\label{eq:Action_free_fermion}
\end{equation}

where $\Delta=\pi\left|V^{2}\right|\rho$, where $\rho$ is the density
of states of the free fermions. the density of states for the impurity
is given by: 
\begin{equation}
\rho_{s}\left(\omega\right)=\frac{1}{\pi}\frac{\Delta}{\Delta^{2}+\left(\omega-E_{f}\right)^{2}}\label{eq:density_states}
\end{equation}

This approach does not capture the Abrikosov-Suhl resonance \cite{key-1}
but provides a good starting point when considering the density of
states near the resonance near $-\left|E_{f}\right|$. We will therefore
keep $\lambda$ general as every $\lambda$ produces a proper meanfield
and study the Abrikoson-Suhl resonance \cite{key-1} using this more
general formulation. 

\section{\label{sec: Kondo model} The Kondo model }

To study the physics of the Abrikoson-Suhl resonance in its simplest
form and to study the soft constraint for another model we will now
switch gears and study the Kondo model which is related to the infinite
$U$ Anderson model \cite{key-1}. The Kondo model is given by the
following Hamiltonian: 
\[
H_{Kondo}=\sum_{k,\sigma}\epsilon_{k}c_{k,\sigma}^{\dagger}c_{k,\sigma}+J\sum_{k}\vec{S}_{k}\cdot\vec{S}_{0}
\]
Where $\vec{S}_{0}$ is a localized spin located at the origin. For
a uniform density of states for the fermions $c_{k,\sigma}$ this
model has an exact solution by the Bethe ansatz leading to exact results
for the magnetic susceptibility, density of states and heat capacity
\cite{key-9,key-8}, we will strive to reproduce these results using
a simpler meanfield method. As a first step let us consider the Read-Newns
formulation of the single impurity Kondo model, with Lagrangian given
by: 
\begin{eqnarray}
L_{RN} & = & \sum_{k,\sigma}c_{k\sigma}^{\dagger}\left(\frac{d}{d\tau}+\epsilon_{k}\right)c_{k\sigma}+\sum_{\sigma}f_{\sigma}^{\dagger}\left(\frac{d}{d\tau}+\lambda_{RN}\right)f_{\sigma}\nonumber \\
 & - & \sum_{k,\sigma}\left\{ \bar{V}\left(c_{k,\sigma}^{\dagger}f_{\sigma}\right)+V\left(f_{\sigma}^{\dagger}c_{k,\sigma}\right)\right\} +2\frac{V\bar{V}}{J}-\lambda_{RN}\,,\label{eq:read_newens_lagrangian}
\end{eqnarray}
where $\lambda_{RN}$ is a multiplier that enforces single occupancy
$\sum_{\sigma}f_{\sigma}^{\dagger}f_{\sigma}=1$ in the spin-1/2 fermion
representation of the magnetic impurity.

At mean field level, the Read-Newns formulation predicts a Kondo temperature
\begin{equation}
T_{K}\sim D\exp\left(-\frac{1}{J\rho}\right)\,,\label{eq:Kondo_meanfield}
\end{equation}
where $D$ is the band width. The exact Kondo temperature however
is given by: 
\begin{equation}
T_{K}\sim D\sqrt{J\rho}\exp\left(-\frac{1}{J\rho}\right)\,,\label{eq:Kondo_exact}
\end{equation}
where the factor of $\sqrt{J\rho}$, missing in the mean field result,
can easily lead to a substantial correction to the value of $T_{K}$.
This discrepancy has measurable consequences, since for instance the
Kondo temperature is directly related to the susceptibility: 
\begin{equation}
\chi\sim\frac{\left(g\mu_{B}\right)^{2}}{T_{K}}\,.\label{eq:Suspetibility}
\end{equation}
Furthermore the meanfield Wilson ratio is one, indicating that the
meanfield also greatly underestimates the heat capacity.

\section{\label{sec: soft constraint} Soft constraint approach }

Here we consider a different approach, which starts by re-writing
the condition $\sum_{\sigma}f_{\sigma}^{\dagger}f_{\sigma}=1$ as
\begin{eqnarray}
\left(1-n_{\uparrow}-n_{\downarrow}\right)^{2}=n_{\uparrow}n_{\downarrow}+(1-n_{\uparrow})(1-n_{\downarrow})=0\,.
\end{eqnarray}
Whereas this approach is identical to the Read-Newns formulation in
the exact Lagrangian, it must be handled differently at mean field
level. Indeed, the constraint is expressed now in terms of a positive
semi-definite function that needs to be set to zero, and this cannot
be achieved at mean field level as it would necessarily give a diverging
mean field parameter when solved self-consistently. We introduce instead
an auxiliary non-interacting fermionic Hilbert space that is constrained
to be trivially empty: $h^{\dagger}h=0$. For any non-integer positive
parameter $K$, we can then combine the constraints in the original
problem and in the auxiliary fermions by imposing: 
\begin{eqnarray}
n_{\uparrow}n_{\downarrow}+(1-n_{\uparrow})(1-n_{\downarrow})-K\,h^{\dagger}h=0\,.
\end{eqnarray}
Notice that this is exactly equivalent to enforcing $n_{\uparrow}n_{\downarrow}+(1-n_{\uparrow})(1-n_{\downarrow})=0$
and $h^{\dagger}h=0$ separately. However, we are now able to treat
the combined constraint at mean field level and look for finite parameters
at saddle point.

The soft constraint Lagrangian can then be written as 
\begin{eqnarray}
L_{SC} & = & \sum_{k,\sigma}c_{k\sigma}^{\dagger}\left(\frac{d}{d\tau}+\epsilon_{k}\right)c_{k\sigma}+\sum_{\sigma}f_{\sigma}^{\dagger}\frac{d}{d\tau}f_{\sigma}\nonumber \\
 & - & \sum_{\sigma}\left\{ \bar{V}\sum_{k}\left(c_{k,\sigma}^{\dagger}f_{\sigma}\right)+V\left(f_{\sigma}^{\dagger}c_{k,\sigma}\right)\right\} +2\frac{V\bar{V}}{J}\nonumber \\
 & + & \lambda_{SC}\left[n_{\uparrow}n_{\downarrow}+(1-n_{\uparrow})(1-n_{\downarrow})-K\,h^{\dagger}h\right]\nonumber \\
 & + & h^{\dagger}\frac{d}{d\tau}h\,.\label{eq: lagrangian soft}
\end{eqnarray}

Notice that Eq.\eqref{eq: lagrangian soft} is not a mean field Lagrangian
because of the terms $\propto n_{\uparrow}n_{\downarrow}$. It is
possible to use the Kotliar Rukenstein formulation of the Anderson
model to convert our Kondo path integral into a mean field \cite{key-12,key-17}.
We recall that the Kotliar Rukenstein slave formulation uses four
slave bosons, $e$, $d$ and $p_{\sigma}$. These represent the empty,
doubly occupied and spin up and spin down singly occupied states.
Immediately one needs to impose the constraint: 
\begin{equation}
e^{\dagger}e+\sum_{\sigma}p_{\sigma}^{\dagger}p_{\sigma}+d^{\dagger}d=1\,,\label{eq:G_R_Constraint-1}
\end{equation}
that there is one physical state. Furthermore to ensure that the state
of the fermion is correlated with the state of the boson, one needs
to impose two constraints, one per spin species: 
\begin{equation}
f_{\sigma}^{\dagger}f_{\sigma}=p_{\sigma}^{\dagger}p_{\sigma}+d^{\dagger}d\,.\label{eq:G_R_FERMI-1}
\end{equation}
The electron operator is then given by: 
\begin{equation}
f_{\sigma}\rightarrow z_{\sigma}f_{\sigma},\qquad z_{\sigma}=e^{\dagger}p_{\sigma}+p_{-\sigma}^{\dagger}d\,.\label{eq:Electron_Kotliar-1}
\end{equation}
It is also conventional to transform 
\begin{equation}
z_{\sigma}\rightarrow\left(1-d^{\dagger}d-p_{\sigma}^{\dagger}p_{\sigma}\right)^{-1/2}z_{\sigma}\left(1-e^{\dagger}e-p_{-\sigma}^{\dagger}p_{-\sigma}\right)^{-1/2}\,.\label{eq:Z_Transform-1}
\end{equation}

The operators $z_{\sigma}$ move between the four allowed boson states
while $f_{\sigma}$ moves between the fermion states, this identity
insures that the states of the bosons and fermions remain correlated
and that the constraint in Eq.\eqref{eq:G_R_FERMI-1} is satisfied.
The Lagrangian for this formulation is given by: 
\begin{eqnarray}
e^{\dagger}\frac{d}{d\tau}e+d^{\dagger}\frac{d}{d\tau}d+\sum_{\sigma}p_{\sigma}^{\dagger}\frac{d}{d\tau}p_{\sigma}+\sum_{\sigma}f_{\sigma}^{\dagger}\frac{d}{d\tau}f_{\sigma}+\nonumber \\
\sum_{\sigma}\lambda_{\sigma}\left(f_{\sigma}^{\dagger}f_{\sigma}-p_{\sigma}^{\dagger}p_{\sigma}-d^{\dagger}d\right)+\sum_{k,\sigma}c_{k,\sigma}^{\dagger}\frac{d}{d\tau}c_{k,\sigma}+\nonumber \\
V\sum_{k,\sigma}c_{k,\sigma}^{\dagger}z_{\sigma}f_{\sigma}+h.c.+\sum\left(\epsilon_{k}-\mu\right)c_{k}^{\dagger}c_{k}+\nonumber \\
+\lambda_{KR}\left(e^{\dagger}e+\sum_{\sigma}p_{\sigma}^{\dagger}p_{\sigma}+d^{\dagger}d-1\right)+\nonumber \\
+2\frac{V\bar{V}}{J}+\lambda_{SC}\left(e^{\dagger}e+d^{\dagger}d-Kh^{\dagger}h\right)\:\,.\label{eq:Kondo_kotliar_Ruckenstein}
\end{eqnarray}

We find that the zero temperature meanfield partition function is
given by: 
\begin{eqnarray}
\frac{1}{\beta}\ln\left(Z\right) & = & \frac{2}{\pi}Im\left[\xi\ln\left(\frac{\xi}{eD}\right)\right]-\frac{2\Delta}{\pi J\rho}-\sum_{\sigma}\lambda\left(p_{\sigma}^{2}+d^{2}\right)\nonumber \\
 &  & -\lambda_{SC}\left(d^{2}+e^{2}-K_{1}h_{1}^{\dagger}h_{1}\right)\nonumber \\
 &  & -\lambda_{KR}\left(\sum_{\sigma}p_{\sigma}^{2}+d^{2}+e^{2}-1\right)\,.\label{eq:Partition_function_kotliar_ruckenstein_double_soft-1}
\end{eqnarray}

Here $\xi=iz^{2}\Delta+\lambda$ and $\lambda_{\uparrow}=\lambda_{\downarrow}=\lambda$.
Furthermore, 
\begin{eqnarray}
z_{\sigma} & = & \left(1-d^{2}-p_{\sigma}^{2}\right)^{-\frac{1}{2}}2d\left(p_{\sigma}+p_{-\sigma}\right)\left(1-d^{2}-p_{-\sigma}^{2}\right)^{-\frac{1}{2}}\nonumber \\
 & = & z_{-\sigma}\equiv z\,.\label{eq:Z_spin_indpenedent}
\end{eqnarray}

The self consistent mean field conditions (together with particle-hole
symmetry) give us that 
\begin{eqnarray}
\frac{K}{2} & = & e^{2}=d^{2}\label{eq:Quadratic_solution}\\
 & = & \frac{4\frac{\Delta}{\pi\rho J}+\lambda_{SC}-\sqrt{\left(4\frac{\Delta}{\pi\rho J}+\lambda_{SC}\right)^{2}-8\lambda_{SC}\frac{\Delta}{\pi\rho J}}}{4\lambda_{SC}}\,.\nonumber 
\end{eqnarray}
We note that for large $\lambda_{SC}\rightarrow\infty$, $K,e^{2},d^{2}\rightarrow0$
and indeed we obtain a good projection onto the spin $1/2$ subspace.

We also get that 
\begin{equation}
p_{\uparrow}=p_{\downarrow}\,,\qquad\qquad p^{2}=\frac{1}{2}-d^{2}\,,\label{eq:Qyadratic_Solution}
\end{equation}
and $z$ given by Eq.~\eqref{eq:Z_spin_indpenedent} can be further
simplified to: 
\begin{equation}
z=4dp=-2\left(p-d\right)^{2}+2p^{2}+2d^{2}=1-2\left(p-d\right)^{2}\,.\label{eq:Zed_simplified}
\end{equation}
Finally, we obtain, $\lambda_{\uparrow},\lambda_{\downarrow}$, $\lambda_{KR}=\frac{1+4d^{2}}{2d^{2}}\frac{\Delta}{\pi\rho J}$
and 
\begin{equation}
\Delta=z^{-2}D\exp\left(-\frac{z^{-2}}{\rho J}\right)\,.\label{eq:delta_kotliar_rukenstein-1}
\end{equation}

We have now obtained a family of mean field solutions as a function
of the parameter $K$ that we introduced in our approach. We can take
advantage of this freedom for instance to best fit one of the known
properties of the Kondo model, for instance $T_{K}=\sqrt{\rho J}D\exp\left(\frac{-1}{\rho J}\right)$,
which we can do analytically perturbatively for small $\rho J$. We
now use the identify $\Delta\sim T_{k}=\sqrt{\rho J}D\exp\left(\frac{-1}{\rho J}\right)$,
$\rho J\ll1$, then we get: 
\begin{equation}
\exp\left(-\frac{z^{-2}}{\rho J}\right)\simeq\sqrt{\rho J}\exp\left(\frac{-1}{\rho J}\right)\,,\label{eq:T_K_KR_U}
\end{equation}
and 
\begin{equation}
z^{-2}\simeq1-\frac{1}{2}\rho J\ln\left(\rho J\right),\quad z\simeq1+\frac{1}{4}\rho J\ln\left(\rho J\right)\,.\label{eq:Z_APPROXIMATE}
\end{equation}
We then get: 
\begin{equation}
d^{2}\simeq\frac{1}{4}-\frac{1}{32}\frac{\lambda_{SC}}{\frac{\Delta}{\pi\rho J}},\quad2\left(p-d\right)^{2}\simeq\frac{1}{16}\left(\frac{\lambda_{SC}}{\frac{\Delta}{\pi\rho J}}\right)^{2}\,.\label{eq:Realtions}
\end{equation}
This means that: 
\begin{equation}
\left(\frac{\lambda_{SC}}{\frac{\Delta}{\pi\rho J}}\right)^{2}=-4\rho J\,\ln\left(\rho J\right)\,,\quad\lambda_{SC}=\frac{2\Delta}{\pi}\sqrt{\frac{4\ln\left(1/\rho J\right)}{\rho J}}\,.\label{eq:U_KR_APPROXIMATE}
\end{equation}
Once we have tuned $K$ to match the Kondo temperature, we can obtain
the heat capacity 
\begin{equation}
C_{v}=2k_{B}^{2}T\frac{\pi^{2}}{3}\frac{1}{z^{2}\Delta}\left(-\ln z+1\right)\simeq2k_{B}^{2}T\frac{\pi^{2}}{3}\frac{1}{z^{2}\Delta}\,,\label{eq:Heat_Capacity_KR}
\end{equation}
and the magnetic susceptibility 
\begin{eqnarray}
\chi & \simeq & \frac{g^{2}\mu_{B}^{2}}{-\pi\lambda_{SC}z^{-2}+2\pi pz^{2}\Delta+3\lambda_{SC}}\nonumber \\
 & \simeq & \frac{g^{2}\mu_{B}^{2}}{2\pi pz^{2}\Delta_{0}}\simeq\frac{g^{2}\mu_{B}^{2}}{\pi\Delta}\,.\label{eq:xi}
\end{eqnarray}
Correspondingly, the Wilson ratio is 
\begin{equation}
w=\frac{2}{\pi}+...\simeq0.64\,,\label{eq:Wilson}
\end{equation}
which is also improved with respect to the conventional Read-Newns
mean field value of one.
\begin{thebibliography}{10}
\bibitem[1]{key-1} P. Coleman, I\textit{ntroduction to Many Body
Physics} (Cambridge University Press, 2015).

\bibitem[2]{key-2} R. Flint, \textit{Symplectic-N in strongly correlated
materials} (Rutgers university, thesis 2010).

\bibitem[3]{key-3} A. Aurebach, \textit{Interacting electrons and
quantum magnetism} (Springer Verlag, New York, 1994).

\bibitem[4]{key-4} P. Fazekas, \textit{Lecture notes on correlation
and magnetism} (World Scientific publishing, Singapore, 1999).

\bibitem[5]{key-5} D. I. Khomskii, \textit{Transition metal compounds}
(Cambridge University press, Cambridge, 2014).

\bibitem[6]{key-6} J. D. Sau, S. Sachdev, arXiv 1311.3298

\bibitem[7]{key-7} T. Giamarchi, \textit{Quantum physics in one dimension}
(Oxford University Press, Oxford, 2003).

\bibitem[8]{key-8} F. Schlottman, Physics Reports \textbf{181}, 1-119
(1989)

\bibitem[9]{key-9} A. C. Hewson, The Kondo problem to heavy fermions
(Cambridge University Press, Cambridge, 1993).

\bibitem[10]{key-10} A. M. A. Vaezi, Slave particle study of strongly
correlated electrons (MIT, thesis 2011).

\bibitem[11]{key-11} R. Fresard, J. Khora, and P. Wolfle, \textit{The
pseudoparticle approach to strongly correlated systems}, in \textit{strongly
correlated systems theoretical methods}, A. Avella eds. (Springer-Verlag,
Berlin, 2012)

\bibitem[12]{key-12} G. Kotliar, and A. E. Rukenstein, Phys. Rev.
Lett. \textbf{57}, 1362 (1986)

\bibitem[13]{key-13} R. K. Parithra, \textit{Statistical mechanics}
(Elsevier Butterworth Heinemann, Amsterdam, 1996)

\bibitem[14]{key-14}P. Korbel, W. Wojcik, A. Klenjberg, J. Spalek,
M. Acquarone and M. Lavagana, Eur. Phys. J. B \textbf{32}, 315 (2003).

\bibitem[15]{key-15} X. G Wen, \textit{Quantum field theory of many-body
systems} (Oxford university press, Oxford, 2004).

\bibitem[16]{key-2-3} P. A. Lee, N. Nagaosa and X. G. Wen, Rev. Mod.
Phys. \textbf{78} (2006).

\bibitem[17]{key-3-2} P. Coleman, \textit{Handbook of Magnetism and
Advanced Magnetic Materials}. Edited by H. Kronmuller and S. Parkin.
Vol 1: Fundamentals and Theory. John Wiley and Sons, 95-148 (2007).

\bibitem[16]{key-16} G. Goldstein, C. Chamon and C. Castelnovo in
preparation.

\bibitem[17]{key-17} R. Fresard, J. Kroha and P. Wolfle,\textit{
The pseudoparticle approach to strongly correlated systems} in \textit{Strongly
correlated systems: theoretical method}s. Edited by A. Avella and
F. Mancini (Springer-Verlag, Berlin, 2012).
\end{thebibliography}

\end{document}
